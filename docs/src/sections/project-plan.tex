%! Author = cspiller
%! Date = 17/11/2022

\thispagestyle{plain}
\newpage
\section{Project Plan}\label{sec:project-plan}

\normalsize

\subsection{Plan Preparation and Communication}\label{subsec:plan-preparation-and-communication}

In order to develop an effective plan for the project, the key deliverables for the project\footnote{As outlined in~\ref{subsec:project-objectives}} were analysed and roughly estimated based upon the perceived complexity of producing that deliverable.
Each deliverable was given a broad estimate of the time this would take in proportion to that complexity, with more complex tasks, and tasks that were prone to delay, given more breathing room in terms of the time allocated to them.
These estimates were then mapped to the overall timescale of the project, and set dates were then applied to the completion of these deliverables in relation to the key milestones of the project, such as submission deadlines.
Subsequently, the broader deliverables were then broken down into subtasks and milestones to provide a more granular view of the project plan.
Text-based artefacts were given dates by which certain chapters needed to be completed, and development tasks were given dates by which certain parts of the application should be built and deployed by.
These subtasks helped to keep track of the project's progress.

This project makes use of the ClickUp\footnote{\citep{clickup}} software to perform project management functions and to organize this author’s workflow.
This software has been chosen over other pieces of software in the education\footnote{\citep{education_software}} and project management\footnote{\citep{pm_software}} space on the basis of cost and ease-of-use, as well as its ability to synchronize across different platforms.
In addition, the platform was also used to communicate the project's progress with the project supervisor who is able to see the project plan and its progress over time through the updating of subtask statuses.

Prioritizing communication with the supervisor was integral to the success of the project because of the accountability and oversight it provided to this author.
Having a platform such as ClickUp drastically reduced the likelihood of error in communication and time management.

\subsection{Timeline}\label{subsec:timeline}

The below series of figures (\ref{fig:timeline1},~\ref{fig:timeline2},~\ref{fig:timeline3}) detail the tasks, subtasks, and the planned timeframes for completion.
Timeframes were adjusted in accordance with the perceived complexity of each task following enlightenment from the literature review and market research.
Dependencies for each task were also calculated and can be seen as arrow representation in the timeline figures.

\begin{figure}[!htb]
    \minipage{\textwidth}
    \includegraphics[width=\linewidth]{timeline1}
    \caption{Timeline: $Dec~12th~\rightarrow~Feb~12th$}\label{fig:timeline1}
    \endminipage\hfill
    \minipage{\textwidth}
    \includegraphics[width=\linewidth]{timeline2}
    \caption{Timeline: $Jan~30th~\rightarrow~Mar~26th$}\label{fig:timeline2}
    \endminipage\hfill
    \minipage{\textwidth}
    \includegraphics[width=\linewidth]{timeline3}
    \caption{Timeline: $Mar~27th~\rightarrow~May~14th$}\label{fig:timeline3}
    \endminipage
\end{figure}

\subsection{Resources}\label{subsec:resources}

This project intended to use minimal resources in its development.
As outlined in~\ref{sec:literature-review}, one of the many advantages of cloud computing is its flexibility and ease of resource management.
Given that the application would be hosted entirely in cloud environments, there would be no hardware costs associated with the project.
The costs that do apply will relate to the use of the~\gls{aws} platform.
These costs needed careful management, as explained in~\ref{subsec:business-risks}.

Other resources utilized included:

\begin{enumerate}
    \item Queen Mary library resources
    \item The project supervisor
    \item Knowledge sharing from colleagues at~\gls{sie}
    \item JetBrains’~\gls{ide} Suite
    \item Open-source audio-processing libraries.
    \item Online articles and tutorials
\end{enumerate}
