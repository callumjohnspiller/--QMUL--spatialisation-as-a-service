\thispagestyle{plain}
\newpage
\section{Testing and Evaluation}\label{sec:testing-and-evaluation}

\normalsize

Testing for the project was a continuous process.
Given the nature of the cloud environment,
components of the system needed to be deployed in an environment so that they could be tested.
Section~\ref{subsec:continuous-integration-continuous-deployment} describes how the project was set up
to automatically run tests on the system's functionality using the~\gls{playwright} framework.
Commits to the GitHub repository for the project triggered redeployment and the ability
to test changes in a cloud environment quickly in addition to running regression tests each time.
However, for the success of the project, manual testing also needed to take place.

\subsection{Manual Testing}\label{subsec:manual-testing}

Manual testing was required to ensure that the functionality of the service was in place.
Oftentimes,
this took the form
of deploying either a~\gls{docker} image
to run in~\gls{aws-lambda} or the~\gls{react} front-end to~\gls{amplify} and
then manually testing the changes that were made.
Based on the results of the manual testing, the next stage of development would be determined,
whether that be a failure in the newly deployed version, or moving onto another feature to develop.

\subsection{User Testing}\label{subsec:user-testing}

In order to identify potential issues and areas of improvement for early versions of the product,
it was important to gather feedback from users who did not have intimate knowledge of the system.

User testing was carried out
by deploying a stable early version of the system to its own~\gls{amplify} environment
and then sending out a link to that deployment along with a form that allowed testers to submit feedback.
This form was made up of questions
that solicited both qualitative and quantitative answers relating to the functionality of the system
and how easy it was to use.
The responses to this form can be found in~\ref{sec:appendix:-user-testing-responses}.

The response gave a gauge not only on the success of the project, but on what further revisions could be made.
All the respondents reported that the project functioned as expected with no failures reported.
Most respondents reported an increase in understanding of what spatial audio was after using the product,
however, some respondents self-reported that after having used the product,
they did not know what was meant by spatial audio, with some confusing it still with stereo audio products.

Many respondents noted
the lack of development in the~\gls{ui} and requested a better understanding
of what aspects of the program were performing which function through the use of more text or labelling.
One user reported a playback bug that was fixed in later versions,
and another requested real-time mixing of the audio signals, as in modern digital audio workstations.
Crucially,
all respondents were unable
to describe a level of access to spatial audio that compared to the one the project offers,
meaning that the project was successful in increasing access to customisable spatial audio.

Requested features from testing:

\begin{itemize}
    \item A graphical user interface (Implemented).
    \item Loading bar for separation progress.
    \item Loading bar for~\gls{stem} playback (Implemented).
    \item Guidance for playing around with the product's sliders.
    \item Live feedback on spatial positions (Implemented).
    \item Synchronised playback of~\glspl{stem} (Implemented).
    \item Mute and solo buttons for each~\gls{stem} during playback (Implemented).
    \item Tidying of layout and formatting (Implemented).
    \item Loading screen when uploading files (Implemented).
\end{itemize}

Having these requests made it easy to make improvements to the product that would improve the~\gls{ux}.
It also resulted in a far more in-depth and refined final product.
