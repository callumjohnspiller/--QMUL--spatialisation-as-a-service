\thispagestyle{plain}
\newpage
\section{Discussion \& Conclusions}\label{sec:discussion-conclusions}

\normalsize

Success based on the initial criteria of the project. Met the core functional requirements and succeeded in creating a cloud-based service that is genuinely unique and allows easy access to trialling high-quality spatial audio to anyone with a browser and a music file.

Additionally, user feedback was very good - most were impressed with the core product despite having reservations about the user interface and its ability to convey information and instructions to the user.

However, the major issue with the project was in its inaccurate user requirements that needed redefining in the face of feedback. Initially was only concerned with the functionality for MVP. However didn't not fully consider the fact that it was real people using the project who may not initially understand what the software did. User experience was far more important than initially thought to user satisfaction. To the extent that some users did not understand the idea of spatial audio and confused the product with stereo music making software.

Given this, I did a good job of pivoting and, once the core functionality of cloud architecture was finished, spent a lot more time refining the user experience. This resulted in: a slideshow explaining the process, a much more refined ui, and even a live mockup of the results, with previews of the individual stems and able to change the parameters and hearths results in real time. While this feature did not have the same level of quality as the 3dti render, it vastly improved the user experience and showed how sources can be moved around in the virtual 3d space.

Overall I am very proud of the project and felt it hit all of the major requirements. I successfully managed to learn a bunch of new pieces of software and libraries, especially the services offered by AWS. I was able to avoid the normal pitfalls of waterfall development by adapting and meeting the need for altered requirements. The final software also showcases a good degree of skill in engineering, with code running efficiently and being well documented, in addition to following code style practises.

Given another few weeks of development I would have done better refactoring and separating the concerns of my front end code, however, given the complexity of the proposed product it is a great achievement to get a finished product. Also able to stay on time and project wasn't derailed even when falling behind (like trips to LA lol)

Possible to perform complex and compute-heavy audio processing activities in a public cloud environment, even including user input to guide the process. Main drawbacks are the fact that processing time can take a while in some activities. For example, tensorflow being used for audio stem separation. However, the project proved that such an architecture is possible and improves accessibility to those without the requisit hardware.

The list of technologies used in the project are:

\begin{itemize}
    \item Python
    \item Spleeter
    \item TypeScript
    \item Node.js
    \item AWS API for Node.js
    \item React.js
    \item MUI
    \item React Three Fiber
    \item Howler.js
    \item C++
    \item AWS API for C++
    \item 3d Tune-In Toolkit
    \item CMake
    \item Docker
    \item AWS Lambda
    \item AWS Step Functions
    \item AWS S3
    \item AWS Event Bridge
    \item AWS SQS
    \item AWS Amplify
    \item Git
    \item LaTeX
\end{itemize}

As a promotional product success was mixed. Significant amount of UX work was required to produce a product that was customer friendly, and in the context of SIEE, this was not given as much consideration as required. Future developments would focus on these issues as well as improving the performance of the overall processing pipeline.