\thispagestyle{plain}
\newpage
\section{Discussion \& Conclusions}\label{sec:discussion-conclusions}

\normalsize

At the conclusion of the project's development,
all the deliverables outlined in~\ref{subsec:project-objectives} have been produced.
A functional web service has been developed
and deployed to a live environment,
and users are able to trial spatial audio using their own music files and input spatial parameters to their liking.

However, despite the project's successes,
there were a number of issues encountered along the way
which suggest ways in which the planning and execution of the project could have been improved.

\subsection{Successes}\label{subsec:successes}

The main success of the project was that a functionally complete product\footnote{Defined in~\ref{sec:requirement-capture}} was produced by the end of the development phase.
Despite a big shift in the requirements of the project after user feedback,
a significant amount of effort was put in to meet these new requirements.
The revising of, and adaption to, new requirements also highlighted the benefits
of successfully adopting an~\gls{agile} development methodology.

The project was also successful in porting local audio processing to a cloud environment,
effectively bringing the benefits of cloud infrastructure\footnote{As outlined in~\ref{subsec:cloud-computing}} to workflows
that are typically performed on a local machine.
This resulted in the successful creation of a~\gls{poc} that
may be adapted
to hosting proprietary software created by~\gls{sie} in order to demonstrate~\gls{ps5} capabilities to~\glspl{pspartner}.
However,
the issue of cloud vendor locking may need to be considered in this case
if the pipeline needs to be ported to another cloud service.

As a learning exercise, the project was extremely effective and required engagement with a wide range of tools.
The project's proposal was complex and ambitious considering the time available.

The exhaustive list of technologies used in the project is as follows:

\begin{itemize}
    \item Python
    \item Spleeter
    \item TypeScript
    \item Node.js
    \item AWS API for Node.js
    \item React.js
    \item Playwright
    \item MUI
    \item React Three Fiber
    \item Howler.js
    \item C++
    \item AWS API for C++
    \item 3d Tune-In Toolkit
    \item CMake
    \item Docker
    \item AWS Lambda
    \item AWS Step Functions
    \item AWS S3
    \item AWS Event Bridge
    \item AWS SQS
    \item AWS Amplify
    \item AWS ECR
    \item Git
    \item LaTeX
\end{itemize}

Such a deep tech stack
allowed for excellent and continuous learning opportunities and the creation of a well-optimized development environment.
The project makes use of all these technologies, and there is little to no redundancy in the libraries and modules used.
Additionally,
the fact
that many of these technologies weren't planned for use at the outset of the project\footnote{As outlined in~\ref{sec:project-plan}},
shows the successes in quickly obtaining the requisite knowledge for the completion of the project.

Feedback from user testing was, on the whole, positive.
Constructive comments from the feedback were taken on board
and were allowed to shape and define new requirements for the project, specifically the project's front-end interface.
Feedback also confirmed that the project met the system requirements and highlighted how it missed user requirements,
meaning that because user testing was performed, the final product was far more successful in meeting its goals.

The new requirements were adjusted to well, and, once the core functionality of cloud architecture was finished,
a significant amount of time was spent refining the user experience.
This resulted in a slideshow explaining the process,
a refined~\gls{ui} featuring a~\gls{3d} representation of the audio spatialisation and even a live mockup of the spatial audio
using the web audio~\gls{api}.
This meant the user could hear previews of the individual stems with spatialisation applied
and be able to change the parameters and hear the results in real time.
While this feature didn't have the same level of quality as the 3DTI Toolkit render,
it vastly improved the user experience and showed how sources can be moved around in the virtual 3d space.
These features were never part of the initial specification.

The project was also successful in highlighting gaps in user's access to spatial audio.
User testing showed that users tended to not have easy access to spatial audio or even understand what it is doing.
The fact that users reported an increase in understanding of spatial audio,
despite the then-underdeveloped user interface, shows the success of the project in demonstrating the technology;
and the fact
that every user who tested the project was able
to produce a final output with just a web browser illustrates the way in which the product increases accessibility.
In the context of~\gls{sie},
this project might be easily sent
to~\glspl{pspartner} who don't have an active understanding of spatial audio,
so they can get a better understanding of the~\gls{ps5}'s technical offering.

In addition, other than the incident described in~\ref{subsubsec:recursion-incident},
costs for the project remained low.
This is particularly important in the business context,
and the fact that the project rarely raised costs over \$0.10 USD speaks to the economic efficacy of public cloud services
in hosting and shaping business solutions.

\subsection{Failures}\label{subsec:failures}

The majority of problems with the project stemmed from the inaccurate user requirements
defined at the outset of the project.
Had the initial specification been more conscious of the user experience from the outset\footnote{See~\ref{subsec:web-interfaces}},
the project's plan might have been scoped more effectively for the development of an effective~\gls{ui}.

Initially, the project was primarily focused on the functionality required for the~\gls{mvp}.
However, the fact that users of the project may not initially understand what the software did,
and what spatial audio was, meant that the overall impression of the product in user testing was insufficient;
user experience was far more important than initially thought.

The results of the user testing\footnote{See~\ref{subsec:user-testing}} meant
that new requirements were drawn up that related to the user interface.
Given that the new requirements added on a significant amount of additional work,
the project's plan didn't have much time remaining in which to accommodate these features,
leading to unwarranted pressure on the development team.

In addition, delays to the project's completion can be traced back to the overall complexity of the project.
While the ambition of the project allowed for a significant amount of learning,
it also meant that the velocity of development was reduced.
The project often fell behind the proposed timescales as seen in~\ref{sec:project-plan}.
This was due, in part, to the unexpected need to learn new technologies such as Docker and React Three Fiber.
However,
the fact that this project was undertaken with respect to the~\gls{agile} methodology meant
that the pivoting to new requirements was expected, and,
due to the iterative development procedures which were set in place,
the new features could be completed on time, if behind schedule.

The project's security might also be called into question.
While the recursion incident was already covered in~\ref{subsubsec:recursion-incident},
there were also two instances of~\gls{aws} credentials being leaked.
The accidental committing of code that contained credentials led to delays
in which the credentials needed to be revoked and cycled,
with further audit and revision being done on the GitHub repo where code was being stored publicly.
Thankfully, notifications had been set up to catch any commits of credentials;
however, the fact this still occurred meant
that more care should have been taken in reviewing commits to any public branches.

Another area for improvement would be the automated testing suite developed as a part of the project.
While unit and end-to-end tests were set up as a part of the~\gls{cicd} pipeline,
the code coverage wasn't as extensive as desired
and meant that certain aspects of the project weren't tested each time they were deployed.
In the future, the tests would be developed fully before writing any new features into the project.

\subsection{Research Questions Revisited}\label{subsec:research-questions2}

To conclude the discussion, the initial research questions should be revisited.

\subsubsection{What are the characteristics of audio-processing pipelines that are executed within cloud infrastructure?}

Audio processing pipelines require a significant amount of preparation to execute in cloud environments.
THis is shown by the need
to package large dependencies for the processing libraries (\ref{subsec:lambda-function-dockerfiles}).
In addition, pipelines require careful orchestration in order to accept user input (\ref{subsec:best-laid-plans}).
In the case of slow-starting processing functions, optimization needs to take place (\ref{subsec:functional-improvements})

\subsubsection{What is the impact of cloud technology in addressing physical hardware and software limitations?}

The nature of cloud services means those users
who don't have access to hardware or software capable of undertaking spatial audio processing,
are able to access these features by using a browser to access~\gls{saas} products (\ref{subsec:cloud-computing}).

\subsubsection{How effective is cloud infrastructure in facilitating the execution of compute-heavy audio pipelines?}

As outlined in~\ref{subsec:user-testing} and~\ref{subsec:successes},
a spatial audio processing pipeline is developed and deployed to a live environment.
This enabled users to manipulate audio into a spatial format using parameters of their choosing.

\subsubsection{How can the leveraging of cloud technology improve the experience of the users of~\glspl{sie}~\gls{pspartner} platform?}

Using cloud technology in the ways
demonstrated in this project means that individual employees of~\gls{pspartner} teams can quickly gain access to demonstrations of~\gls{ps5} capabilities without going through the arduous process
of obtaining a~\gls{gdk}.
In addition,
those not intimately familiar with the theory behind specific aspects of~\gls{sie} products can improve their understanding
using a~\gls{saas} product that is accessible via web browser.

\subsection{Personal Thoughts}\label{subsec:personal-thoughts}

Overall I'm very proud of the project and felt it accomplished the goals I had set out for it.
I managed to use the project to expand my knowledge of both audio processing and cloud technology,
which are areas of personal interest for me.

I managed to avoid the normal pitfalls of waterfall development by adapting to,
and meeting the need for, altered requirements.
The final code repository showcases a good degree of skill in software engineering;
code runs efficiently and is well documented.

That being said, given some more time to develop the project, I would have done better
refactoring of the code and made sure it followed code conventions better.
However, given the complexity of the proposed product,
I consider it an achievement to get it to a finished state in the proposed timeframe.

I'm also happy with the way I was able to catch up when the project fell behind schedule.
Unexpected events\footnote{Such as last-minute work trips to LA} did not completely derail the project.

Most importantly, I had a lot of fun doing it!