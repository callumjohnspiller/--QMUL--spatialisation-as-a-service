\thispagestyle{plain}
\newpage
\section{Conclusions}\label{sec:conclusions}

\normalsize

Possible to perform complex and compute-heavy audio processing activities in a public cloud environment, even including user input to guide the process. Main drawbacks are the fact that processing time can take a while in some activities. For example, tensorflow being used for audio stem separation. However, the project proved that such an architecture is possible and improves accessibility to those without the requisit hardware.

As a promotional product success was mixed. Significant amount of UX work was required to produce a product that was customer friendly, and in the context of SIEE, this was not given as much consideration as required. Future developments would focus on these issues as well as improving the performance of the overall processing pipeline.