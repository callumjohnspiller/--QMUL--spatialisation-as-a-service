\thispagestyle{plain}
\newpage
\section{Requirement Capture and Analysis}\label{sec:requirement-capture}

\normalsize

According to~\citet{anton2003successful}: \textquote{good requirements planning means software can be cheaper to produce, easier to build, and less prone to unexpected failure.}
Requirements for this project were gathered through an analysis of scenarios~\citep{potts1994inquiry} and through considering the business, user, and functional requirements these scenarios touch upon~\citep{wiegers2000karl}.
These elements will be combined to produce the software requirements specification.


\subsection{User Personas}

Persona of game developer:
want to experience what spatial audio can do therefore the project must be able
to take user input and render spatial audio based on that inputs as an MVP.

Persona of audio enthusiast:not necessarily a game Dev but someone who is interested in audio development who might like to try playing around with different spatial audio API without setting anything up locally.

Persona of the new user: Never experienced spatial audio, may not even know what it is.
Using the product and expecting to know more about what it means, not just what is available to developers.

Ultimately a rendering pipeline was needed in AWS. If the software can perform all audio processing in the cloud and deliver to frontend, that was MVP.

The project had other stretch goals: have a user interface that allows the user to input values; expand the functionality by specifying HRTF files and SOFA environment files; allow for real-time previewing.

These other goals were defined
by asking potential users + also experts in the field of audio processing in addition to the literature review.

Often more requirements came out in the form of user feedback.
Once people saw an early build, they came up with other `nice to have' features and requirements that went beyond the MVP\@.

MVP reqs:

Functional:
take user input.
separate and render audio in the cloud using HRTF profiles.
deliver output to frontend

Non-functional:
a UI to enter variables.
an elucidation of what spatial audio means: meet the intended stakeholder's requirements.
a 'good enough' delivery time - not so long that the user thinks it has failed

Stretch goal reqs:

Functional
- Preview spatial rendering of individual stems
- Real-time views of spatial audio
-- security - files are managed and secured to protect user info
- fast response time in rendering
- visual representation of the spatial audio rendering in 3d space
- allow user to upload and choose SOFA and BRIR files
- allow multiple types of audio file upload

Non functional
-- fully explains spatial audio and how it works
- a smooth looking ui - take it beyond basic to a potentially customer-facing product
- reliable - minimal bugs in the programming that result in failure for the end user.
- edge case handling


