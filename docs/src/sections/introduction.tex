%! Author = cspiller
%! Date = 17/11/2022
\thispagestyle{plain}
\newpage
\section{Introduction}\label{sec:introduction}
\subsection{Project Background}\label{subsec:project-background}
\normalsize

This project has been undertaken as a part of a degree apprenticeship with~\gls{sie}.
As such, the final year engineering project for this author will consider the business domain of the sponsoring company.

With the release of the~\gls{ps5} in November 2020,~\gls{spatial_audio} technology has come into focus for~\gls{sie} as game developers seek to leverage the~\gls{tempest_3d_audio} espoused by the video game console.
In addition,~\gls{sie} is undergoing a shift to cloud-based infrastructure\footnote{like many other companies relying on web technology~\citep{cc_overview}} across the wider business in order to leverage the cost, flexibility, and reliability benefits that cloud technology affords~\citep{cc_overview}.

With respect to these developments in the business, this project seeks to engage both of these emergent technologies in order to address one of~\gls{sie}\textquotesingle s major~\glspl{sla} with a software solution.

\subsection{Problem Statement}\label{subsec:problem-statement}

The onboarding of~\glspl{pspartner} to the developer and publisher platforms managed by~\gls{sie} is an area in the company that has been targeted for improvement.
One of the primary targets for improvement has been the reduction of friction in the process of getting~\glspl{pspartner} engaged with the PlayStation ecosystem.

While the~\gls{pspartner} onboarding process has been automated and refined over time, this author argues that the ability for potential~\glspl{pspartner} to engage with and trial PlayStation's spatial audio technology is too restricted.

This project was founded upon the hypothesis that, for a user who wishes to experiment with or experience~\gls{spatial_audio}, the barrier for entry is too high.
Engaging with a working demonstration of customisable~\gls{spatial_audio} requires a user to perform a significant amount of preparation and setup on a local machine;
it demands a time commitment and pre-requisite technical knowledge that is far greater than is reasonable for an interested party to quickly evaluate what is possible.

Currently,~\glspl{pspartner} who want to experiment with~\gls{tempest_3d_audio} must apply for and order a~\gls{ps5} development kit, wait for it to arrive, perform the setup of the development kit, then figure out how to engage with the~\glspl{api} provided by~\glspl{sie}~\gls{devnet}.
This process is time-consuming and sub-optimal for a~\gls{pspartner} who wishes to get a quick insight into what is possible.

This project proposes an alternative solution where a~\gls{pspartner} who wishes to engage with the~\gls{spatial_audio} paradigm is able to experiment with the technology by using a familiar audio file of their choosing.
The fact that the system will be accessible through a browser will mean that the~\gls{pspartner} will have much easier access, increasing the overall positive impression of the~\gls{pspartner} platform.

\subsection{Project Aims}\label{subsec:aims}

The aim of this project is to research, design, and engineer a web service that allows a user to easily experience~\gls{spatial_audio} in a way that reacts to their input.
Despite the fact that this project will make use of \gls{dsp} and requires a front-end for the user to interact with, the primary complexity of this project concerns the cloud infrastructure design and implementation that forms the backbone of the\textit{\textquote{Spatialisation-as-a-Service}}~concept, that is, moving compute-heavy tasks from the local machine to a cloud environment.
Taking this into account, the project should be classified as an infrastructure project, and will focus upon the challenges relating to this area.

The proposed workflow would allow a user to use a standard web browser\footnote{The project will most likely be designed for interaction through Chromium-based browsers and Firefox} to upload an audio file to a webpage and then receive back a new audio file that has rendered the stems from the original file into a \textquotesingle spatialised\textquotesingle ~format informed by \glspl{hrtf}.
All the audio processing would execute in \gls{aws} cloud environments to circumvent real-world hardware requirements and challenges.

For information security reasons, the prototype produced as a part of this project will \textit{not} feature any proprietary~\gls{sie} software and instead use only libraries and code that exist in the public domain.
Because of this, the prototype can be considered a~\gls{poc} where, if successful,~\gls{sie} software might be transplanted into the serverless pipeline.

\subsection{Project Objectives}\label{subsec:project-objectives}

In order to achieve the aims set out in~\ref{subsec:aims} the project must produce a number of deliverables:

\begin{enumerate}
    \item A project plan which outlines the timeline of both the research and development of the project
    \item A review of pertinent literature relating primarily to:
    \begin{itemize}
        \item Public Cloud infrastructure and services (especially those available from \gls{aws})
        \item Audio spatialisation
        \item Web front-end technology and audio \glspl{api}
    \end{itemize}
    \item A review of existing stereo-to-spatial services and technologies.
    \item A functioning stereo-to-spatial serverless pipeline.
    \item A frontend that enables a user to interact with the conversion pipeline by uploading and downloading audio files, as well as setting parameters for conversion through a web~\gls{api}.
    \item A testing framework that supports iterative development.
    \item A report on user testing.
    \item A review and analysis of how the produced system has met or missed the targets.
\end{enumerate}

\subsection{Research Questions}\label{subsec:research-questions}

In order to guide the research and development process of the proposed system, this report will seek to answer the following research questions:

\begin{enumerate}
    \item What are the characteristics of audio-processing pipelines that are executed within cloud infrastructure?
    \item What is the impact of cloud technology in addressing physical hardware limitations?
    \item How effective is cloud infrastructure in facilitating the execution of compute-heavy audio pipelines?
    \item How can the leveraging of cloud technology improve the experience of the users of~\glspl{sie}~\gls{pspartner} platform?

\end{enumerate}