%! Author = cspiller
%! Date = 17/11/2022
\thispagestyle{plain}
\newpage
\section{Introduction}\label{sec:introduction}
\subsection{Project Background}\label{subsec:project-background}
\normalsize

This project forms part of a degree apprenticeship hosted by~\gls{sie}.
Because of this, the business domain of the sponsoring company is considered.

With the release of the~\gls{ps5} in November 2020,~\gls{spatial_audio} technology has become a key product for~\gls{sie} as game developers seek to leverage the~\gls{tempest_3d_audio} espoused by the video game console.
In addition,~\gls{sie} is shifting to cloud-based infrastructure\footnote{like many other companies relying on web technology~\citep{cc_overview}} across the wider business in order to leverage the cost, flexibility, and reliability benefits that cloud technology affords~\citep{cc_overview}.

In response to these changes within the company, this project aims to engage both of these emergent technologies in order to address one of~\gls{sie}'s major~\glspl{sla} with a software solution.

\subsection{Problem Statement}\label{subsec:problem-statement}

At the time of writing,
the onboarding of~\glspl{pspartner} to the developer and publisher platforms
managed by~\gls{sie} is an area that has been targeted for improvement in the company.
Getting~\glspl{pspartner} engaged with PlayStation's development and sales ecosystem faster and with less `friction' has been outlined as a major~\gls{sla}.

This report argues that, despite the fact that the~\gls{pspartner} onboarding process has been improved, the ability for~\glspl{pspartner} to trial PlayStation's spatial audio technology is too restricted.
There is currently a dependence on local hardware setup in order to experience PlayStation's~\gls{spatial_audio} in an interactive manner.

Currently~\glspl{pspartner} who want to experiment with~\gls{tempest_3d_audio} must perform the following steps:

\begin{enumerate}
    \item Apply for and order a~\gls{ps5}~\gls{gdk}
    \item Await delivery of the kit
    \item Set up the~\gls{gdk}
    \item Research the ~\glspl{api} documentation provided by~\gls{sie}'s~\gls{devnet}
\end{enumerate}

This process is sub-optimal for a~\gls{pspartner} who wishes to get a quick insight into what is possible with~\gls{spatial_audio}.

This project proposes an alternative solution where~\gls{spatial_audio} can be accessible as a web service,
effectively eliminating the need for a~\gls{gdk} to test a certain area of~\gls{ps5} functionality.
The system should be accessible through a browser, meaning that the~\gls{pspartner} will be granted easier access, fulfilling the aforementioned~\gls{sla}.

\subsection{Project Aims}\label{subsec:aims}

The aim of this project is to research, design, and engineer a web service that allows a user to experience~\gls{spatial_audio} in a way that reacts to their input.
The intended complexity of this project concerns the cloud infrastructure design and implementation that forms the backbone of the\textit{\textquote{Spatialisation-as-a-Service}}~concept;
the project seeks to move compute-heavy~\gls{dsp} tasks from local hardware to a cloud environment.
The project can therefore be classified primarily as an infrastructure project, and discusses the topics and challenges relating to this area.
The project has also been extended to consider issues of~\gls{ux} and how it influenced by~\glspl{ui}.

The proposed workflow allows a user
to upload a music file\footnote{Of either of the .mp3 or .wav format} of their choosing to a website using a standard web browser\footnote{Chromium-based browsers and Firefox}.
The user will ultimately receive back a new audio file rendered by the service that demonstrates~\gls{spatial_audio} as a customised transformation of the original file.
This new audio file will be constructed from the isolated `stems'\footnote{For example, we might separate a song into vocals, bass, and drums} of the original file which will be separated out from each other as a part of the~\gls{dsp} pipeline.
The user will then choose where to `position' these stems in a virtual~\gls{3d} space by setting spatial parameters for each stem.
Finally, the rendering of the stems and their spatial parameters will take place using~\glspl{hrtf} to produce a `spatialised' version of the music file.
A goal of the project is to circumvent any hardware requirements the~\gls{dsp} pipeline may have;
as such, all the audio processing will be executed in~\gls{aws} cloud environments.

For information security reasons, the prototype produced as a part of this project will~\textit{not} feature any proprietary~\gls{sie} software and use only libraries and code that exist in the public domain.
Because of this, the prototype can be considered a~\gls{poc} where, if successful,~\gls{sie} software might be transplanted into the serverless~\gls{dsp} pipeline.

\subsection{Project Objectives}\label{subsec:project-objectives}

In order to achieve the aims set out in~\ref{subsec:aims} the project must produce a number of deliverables:

\begin{enumerate}
    \item A project plan which outlines the timeline of both the research and development of the project
    \item A review of pertinent literature relating primarily to:
    \begin{itemize}
        \item Public Cloud infrastructure and services (especially those available from~\gls{aws})
        \item Audio spatialisation
        \item Web audio~\glspl{api} and front-end technology
    \end{itemize}
    \item A review of existing stereo-to-spatial services and technologies
    \item A functioning stereo-to-spatial serverless pipeline
    \item A frontend that enables a user to interact with the conversion pipeline by uploading and downloading audio files, as well as setting parameters for conversion through a web~\gls{api}
    \item A testing framework that supports iterative development
    \item A report on user testing
    \item A review and analysis of how the produced system has met or missed targets
\end{enumerate}

\subsection{Research Questions}\label{subsec:research-questions}

In order to guide the research and development process of the proposed system, this report will seek to answer the following research questions:

\begin{enumerate}
    \item What are the characteristics of audio-processing pipelines that are executed within cloud infrastructure?
    \item What is the impact of cloud technology in addressing physical hardware limitations?
    \item How effective is cloud infrastructure in facilitating the execution of compute-heavy audio pipelines?
    \item How can the leveraging of cloud technology improve the experience of the users of~\glspl{sie}~\gls{pspartner} platform?
\end{enumerate}