\thispagestyle{plain}
\newpage
\section{Infrastructure \& Design}\label{sec:infrastructure-design}

\normalsize

Figure~\ref{fig:preliminary-design} details the initial design for the project's cloud architecture.
The diagram provides a high-level overview of the primary user journey,
and displays the~\gls{aws} products that were intended to fulfil this journey.
However, given that this project is executed using elements of the~\gls{agile} methodology,
this design was subject to change as development progressed.
This section aims to describe how the initial design evolved
to accommodate the changing system requirements of the project.

\begin{figure}[!htb]
    \minipage{\textwidth}
    \includegraphics[width=\linewidth]{initial_architecture_diagram}
    \caption{Preliminary Design for Cloud Architecture}\label{fig:preliminary-design}
    \endminipage\hfill
\end{figure}

\subsection{Best Laid Plans}\label{subsec:best-laid-plans}

The product is designed for a wide range of users with varying knowledge of~\gls{spatial_audio}.
In order to accommodate as many potential users as possible, a web-hosted~\gls{ui} is considered essential.
The user's primary mode of interaction with the service would be through a website
built using the~\gls{react} front-end library.
This library was chosen because of its ubiquity in modern web engineering, the depth of documentation available,
and existing familiarity with the framework.

The front-end was designed to make use of the~\gls{aws}~\gls{sdk} for~\gls{nodejs}.
This would allow the interface to initiate and interact with the audio processing pipeline hosted in~\gls{aws}.

Figure~\ref{fig:preliminary-design} displays the two~\gls{aws-lambda} functions
that would comprise the backbone of the audio processing pipeline.
Furthermore,
these~\gls{aws-lambda} functions would be coordinated and executed using the~\gls{aws-step-function} service.
This service was highly desired for this purpose
since the processing pipeline would follow a series of discrete steps that require input from the user.
\glspl{aws-step-function} would be able to accommodate this easily.

The entire service was initially intended
to be coordinated solely through the use of the internal messaging service:~\gls{aws-eventbridge}.
This service, in conjunction with~\gls{aws-api-gateway},
was intended to form most of the communication between the back and front ends of the application.

\subsection{Refinement}\label{subsec:refinement}

Problems with the initial designs were highlighted as the project progressed.

The first problems arose with the development of the~\gls{aws-lambda} functions that would be used to process the audio.
\gls{aws-lambda} functions
require a deployment package
to be created\footnote{In the case of the spatialisation lambda function, the program also needed to be compiled, which was done with the use of a CMake build file that can be found in~\ref{lst:cmake}}
and then uploaded to~\gls{aws} in order for the function to have the resources it needs to run.
A fact that was not known previously is that
there is a limit of 50 MB on the size of the deployment packages that could be uploaded directly to~\gls{aws-lambda}.
Given that the 3D Tune-In Toolkit and the Spleeter source separation libraries have significant dependencies
that are more than 50 MB when zipped, a re-design needed to occur.

This issue was solved by making use of the~\gls{aws-ecr} service.
This service allows a developer to host and deploy~\glspl{container-image},
crucially, with a size limit of 75 GB allowed for each image uploaded to the registry.
Through the use of~\gls{docker} container images,
the code for the Lambda functions was bundled into deployment packages
that could be uploaded to an
\gls{aws-ecr} and then linked to a Lambda function
while staying well under the data limit offered by the container registry.
The Dockerfiles for the source separation and spatialisation lambda functions can be found in Appendix~\ref{subsec:lambda-function-dockerfiles}.
Note the installation of the AWS Lambda runtime environment for both Python in Listing~\ref{lst:source-sep-dockerfile} and C++ in Listing~\ref{lst:spatialisation-dockerfile}.

The next major revision to the architecture design came with the removal of the proposed~\gls{aws-api-gateway} interface.
Since the~\gls{aws-step-function} requires input from the user to execute,
the step function callback feature would be used to wait for an input message from the front end to the step function.
The most elegant solution for this problem was
to create a~\gls{aws-sqs} queue through which messages could be sent
and retrieved asynchronously without the risk of losing information through API communication failure.
This was implemented through the creation of a unique message queue per interaction with the product.
A unique identification key is created in the~\gls{react} interface, which is then used
to identify both the~\gls{aws-sqs} queue and the resources uploaded and retrieved from the relevant~\gls{s3} buckets.
This key is not the same as a session key
and would regenerate if the user refreshed the page to return to the start of the application.

With these changes, it became possible to define a final cloud architecture and workflow.

\subsection{Finalising the structure}\label{subsec:finalising-the-structure}

\begin{figure}[!htb]
    \minipage{\textwidth}
    \includesvg[width=\linewidth]{saas_architecture_2.0}
    \caption{Final cloud architecture diagram}\label{fig:final_design}
    \endminipage\hfill
\end{figure}

Figure~\ref{fig:final_design} depicts the final infrastructure design of the project.
The figure has been annotated with numbers to show the order in which data flows around the architecture.
In addition to the changes mentioned above, there are a few other notable changes that are evidenced in the figure.

\begin{itemize}
    \item The~\gls{aws-step-function} has been fleshed out with the discrete steps of the workflow orchestration.
    An Amazon States Language representation of the function can be found in Listing~\ref{lst:step-function-json}.
    \item There are three~\gls{s3} buckets to hold the artefacts from each stage of the process.
    \item The~\gls{react} frontend is now hosted in~\gls{amplify} instead of~\gls{ec2};
    allowing for the implementation of~\gls{cicd} each time a commit is pushed to a branch in GitHub.
\end{itemize}

\subsection{A fresh coat of paint}\label{subsec:a-fresh-coat-of-paint}

Early user testing of the product revealed
that elements of the front-end interface needed improving\footnote{Further information on the testing and feedback process can be found in Section~\ref{sec:testing-and-evaluation}}.
The initial interface was basic with minimal styling applied to the~\gls{react} interface.
See Figure~\ref{fig:early-fe} for a screenshot of the interface's early form.

\begin{figure}[!htb]
    \minipage{\textwidth}
    \includegraphics[width=\linewidth]{early-ui}
    \caption{Early implementation of application frontend}\label{fig:early-fe}
    \endminipage\hfill
\end{figure}

After the~\gls{mvp} was achieved, there was some additional capacity available to improve the frontend.

\begin{itemize}
    \item The project was adapted to use tools provided by~\gls{mui},
    a~\gls{react} component library\footnote{https://mui.com/core/},
    in order to better structure the project and improve the aesthetic feel.
    See Figure~\ref{fig:ui-splash} for the improved welcome screen for the application.
    \item A slideshow was developed using the component library that explained the theory behind spatial audio in an easy-to-understand manner and explains what the product aims to do (Figure~\ref{fig:ui-slideshow}).
    \item The music file upload stage of the application was redesigned using~\gls{mui} and turned into a three-step process using the `Stepper' component (Figure~\ref{fig:ui-uploader}).
    \item The `Spinner' component is used to help decorate the process of waiting for the~\glspl{stem} to be separated out (Figure~\ref{fig:ui-spinner}).
\end{itemize}

\begin{figure}[!htb]
    \minipage{\textwidth}
    \includegraphics[width=\linewidth]{ui-splash}
    \caption{Refined splash screen that greets the user}\label{fig:ui-splash}
    \endminipage\hfill
\end{figure}

\begin{figure}[!htb]
    \minipage{\textwidth}
    \includegraphics[width=\linewidth]{ui-slideshow}
    \caption{One of the slides from the introductory explanation of the project}\label{fig:ui-slideshow}
    \endminipage\hfill
\end{figure}

\begin{figure}[!htb]
    \minipage{\textwidth}
    \includegraphics[width=\linewidth]{ui-uploader}
    \caption{Audio upload stage remodelled using the~\gls{mui} Stepper component}\label{fig:ui-uploader}
    \endminipage\hfill
\end{figure}

\begin{figure}[!htb]
    \minipage{\textwidth}
    \includegraphics[width=\linewidth]{ui-spinner}
    \caption{\gls{mui} Spinner and message displayed while audio~\glspl{stem} are separated}\label{fig:ui-spinner}
    \endminipage
\end{figure}

\subsection{No spatialisation without representation}\label{subsec:no-spatialisation-without-representation}
An important piece of feedback on early demos of the product was that there was not enough visual or audio feedback on the user's input of the spatial parameters.
As a result of this,
the most significant redesign of the product allowed the user
to experience and preview the effects of their input
instead of being forced to wait for the final product to render and be delivered to the frontend.
Using React Three Fiber, a~\gls{react} renderer for~\gls{threejs}\footnote{\citep{dirksen2023learn}},
a 3D model represents the positioning of the audio~\glspl{stem} in the virtual spatial field
that the application eventually renders.
The input fields for each~\gls{stem}'s spatial positions are now overlaid on top of the model.
Previously only lists of sliders were shown for each~\gls{stem},
but the final version lets
the user see and hear what changes are being made in real-time as the~\glspl{stem} are being played back.
Figures~\ref{fig:ui-3d-1},~\ref{fig:ui-3d-2},
and~\ref{fig:ui-3d-3} demonstrate how the 3D model displays the user's input on screen.
The user is able
to rotate and zoom in and out of the model
to see where the~\glspl{stem} are positioned in relation to the listening position at the origin of the model.
In addition, instead of the~\glspl{stem} being played separately as before, the~\glspl{stem} synchronise to each other,
and the user can mute or `solo' each~\gls{stem} as they wish.
This functionality is similar to that found in modern Digital Audio Workstations.
The Howler.js library,
which is used for audio playback, allows the accessing of the spatial audio module in the Web Audio API\@.
Because of this, the user can preview the audible position of each~\gls{stem} while they are playing in the application.
However,
the quality of this preview is not as high-quality as the final rendered output that is created using the 3D-TuneIn API\@;
this justifies the retention of the option to render the audio using the workflow hosted in~\gls{aws-lambda}.

\begin{figure}[!htb]
    \minipage{\textwidth}
    \includegraphics[width=\linewidth]{ui-3d-1}
    \caption{Perspective 1 of the 3D model interface}\label{fig:ui-3d-1}
    \endminipage
\end{figure}

\begin{figure}[!htb]
    \minipage{\textwidth}
    \includegraphics[width=\linewidth]{ui-3d-2}
    \caption{Perspective 2 of the 3D model interface}\label{fig:ui-3d-2}
    \endminipage
\end{figure}

\begin{figure}[!htb]
    \minipage{\textwidth}
    \includegraphics[width=\linewidth]{ui-3d-3}
    \caption{Perspective 3 of the 3D model interface}\label{fig:ui-3d-3}
    \endminipage
\end{figure}