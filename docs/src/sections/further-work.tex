\thispagestyle{plain}
\newpage
\section{Further Work}\label{sec:further-work}

\normalsize

Thankfully this project met a lot of the requirements, even the ones found after user testing.

However improvements could be made to the cold starts of the cloud rendering
Initial separation of stems takes a long time because of cold starts and a way to reduce this would be most important could also add a load bar to track the progress of the lambda function would go a long way to I creasing user experience metrics

Demonstrate the capabilities of 3d toolkit better - allow changing of HRTF and BRIR profiles to the users tastes - this could be reflected in the 3d model also - it's good for basic demo but going a bit further would be great

Finally; improve UI further - make more detailed and precise.

The project had other stretch goals: have a user interface that allows the user to input values; expand the functionality by specifying HRTF files and SOFA environment files; allow for real-time previewing.

Stretch goal reqs:

Functional
- Preview spatial rendering of individual stems
- Real-time views of spatial audio
-- security - files are managed and secured to protect user info
- fast response time in rendering
- visual representation of the spatial audio rendering in 3d space
- allow user to upload and choose SOFA and BRIR files
- allow multiple types of audio file upload

Non functional
-- fully explains spatial audio and how it works
- a smooth looking ui - take it beyond basic to a potentially customer-facing product
- reliable - minimal bugs in the programming that result in failure for the end user.
- edge case handling