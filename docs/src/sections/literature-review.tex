%! Author = cspiller
%! Date = 17/11/2022

\thispagestyle{plain}
\newpage
\section{Literature Review}\label{sec:literature-review}

\normalsize

As outlined in~\ref{subsec:aims}, this project attempts to engage with cloud infrastructure, spatial audio processing, and web development frameworks.
To approach these topics thoroughly, this report performs a review of pertinent literature.
In doing so, this report considers and incorporates existing ideas in these fields while providing a strong foundation to answer the research questions listed in~\ref{subsec:research-questions}.

\subsection{Audio Spatialisation}\label{subsec:audio-spatialisation}
\subsubsection{Origins}

\citet{blauert_spatial} notes in their seminal text,~\textit{\usebibentry{blauert_spatial}{title}}, that: \textquote{human beings are primarily visually-orientated}, and that the other senses are less developed in comparison.
This difference has been mirrored within the literature;
\citet{wade_binaural} note that, over the course of history, research in binaural hearing is less developed than that of binocular vision.
The concept of distinction between the~\textit{sound event} and the~\textit{auditory event} as influenced by binaural hearing is therefore a relatively modern development~\footnote{\citep{blauert_spatial}} and informs the practice of audio replication analogous to its originating sound event:

\begin{quotation}~
    The telecommunications engineer, of course, is especially interested in just those cases in which the positions of the sound source, and the auditory event do not coincide.
    The telecommunications engineer seeks to reproduce the auditory events that occur at the point where a recording or transmission originates, using the smallest possible number of sound sources (e.g., loudspeakers)~\citep{blauert_spatial}.
\end{quotation}

The patent filed by Alan Dower Blumlein~\footnote{\citep{blumlein-patent}} exploits human binaural hearing so that entertainment experiences might be enhanced.
Blumlein observed that in film theatres there was a certain level of cognitive dissonance whereby the actor’s voices sounded like they were coming from a different location than where they appeared on the screen~\citep{alexander_blumlein}.
Blumlein recognized that humans can determine where another human voice in a room is located as a result of binaural hearing, and so his patent explores how sound might be recorded or played back in a way that induces an auditory event exhibiting spatialisation on the horizontal plane~\citep{blumlein-patent}.

The patent marked an improvement in the way that auditory events might be replicated by introducing this form of spatialisation, and the vestiges of Blumlein`s ideas can be observed in modern spatial audio techniques~\citep{politis_spatial, beyer_acoustics}.
What is more salient, however, is that this patent also shows how the human’s ability to locate sound sources is not solely dependent on the nature of the sound itself, as identified by~\citet{blauert_spatial}.
Rather, it is the case that many other physiological and psychological factors are at play.

\subsubsection{From Two To Three}

