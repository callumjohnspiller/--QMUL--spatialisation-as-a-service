%! Author = cspiller
%! Date = 17/11/2022

\thispagestyle{plain}
\newpage
\section{Literature Review}\label{sec:literature-review}

\normalsize

As outlined in~\ref{subsec:aims}, this project attempts to engage with cloud infrastructure, spatial audio processing, and web development frameworks.
To approach these topics thoroughly, this report performs a review of pertinent literature.
In doing so, this report considers and incorporates existing ideas in these fields while providing a foundation from which to answer the research questions listed in~\ref{subsec:research-questions}.

\subsection{Cloud Computing}\label{subsec:cloud-computing}

Ever since internet service providers began the commercialisation of cloud computing, it has become one of the major trends in the technology space~\citep{cc_overview}.
According to~\citet{cc_overview}, `cloud computing' is one of the most vague terms when it comes to the description of the technology on account of the breadth of its application.

Cloud computing technology is dominated by three major players, each with their own style of cloud services:
\begin{enumerate}
    \item Amazon's Web Services which began as a means of server virtualization~\citep{awsintro}
    \item Google's Cloud Platform, described by~\citet{cc_overview} as a technique-specific sandbox that calls itself a~\gls{paas}~\citep{googlecloudintro}
    \item Microsoft's Azure Network~\citep{azurefundamentals}
\end{enumerate}

% TODO Continue from here

s2\citep{cc_challenges}

\subsection{Audio Spatialisation}\label{subsec:audio-spatialisation}
\subsubsection{Origins}

\citet{blauert_spatial} notes in their seminal text,~\textit{\usebibentry{blauert_spatial}{title}}, that: \textquote{human beings are primarily visually-orientated}, and that the other senses are less developed in comparison.
This difference has been mirrored in the history of scientific research.
\citet{wade_binaural} note that research in binaural hearing was developed later than binocular vision partially due to the difficulty in controlling the audio stimuli in experiments.
It was only later on that the concept of distinction between the~\textit{sound event} and the~\textit{auditory event} as influenced by binaural hearing became prevalent.
\citet{blauert_spatial} explains that this distinction informs the practice of audio replication analogous to its originating sound event:

\begin{quotation}
    The telecommunications engineer, of course, is especially interested in just those cases in which the positions of the sound source, and the auditory event do not coincide.
    The telecommunications engineer seeks to reproduce the auditory events that occur at the point where a recording or transmission originates, using the smallest possible number of sound sources (e.g., loudspeakers)~\citep{blauert_spatial}.
\end{quotation}

The patent filed in 1958 by Alan Dower Blumlein\footnote{\citep{blumlein-patent}} details an early stereophonic system, which exploits the human sound localization ability for the enhancement of entertainment experiences\footnote{This author acknowledges that this is not the first example of this kind of system; the control of inter-aural time differences was pioneered by Cl\'ement Ader as early at 4 years after Bell's invention of the telephone. This was for the purpose of rendering a spatial transmission of the Paris Opera. This further solidifies a history of the desire for spatial immersion in entertainment.}~\footnote{There is a rich history of considering space in the composition of music in Western Classical tradition, with Italian renaissance composers writing for \textit{cori spezzati}, or multiple choirs that are spatially separated~\citep{spezzati}. This author mourns that the topic of spatialisation in historic acoustic performance goes beyond the scope of this report.}.
Blumlein observed that in film theatres there was a certain level of cognitive dissonance whereby the actor’s voices sounded like they were coming from a different location than where they appeared on the screen~\citep{alexander_blumlein}.
This patent, in response, specifically outlines methods for introducing stereophonic audio to sound film as a means of increasing the perceived~\textquote{quality} of the entertainment experience.
Blumlein acknowledges that human binaural hearing is responsible for the ability to localize sound, and his patent is an example of how one might induce an auditory event that exhibits spatialisation on the horizontal plane through the control of inter-aural time differences~\citep{blumlein-patent}.

The patent marked an improvement in the way that auditory events might be replicated by introducing this form of spatialisation, and the vestiges of Blumlein`s ideas can be observed in modern spatial audio techniques~\citep{spatial_techniques, beyer_acoustics}.
What is perhaps the most salient aspect of the document, however, is that it recognizes the physiological factors that are involved in human sound localization.
These physiological factors explored and expounded upon by~\citet{blauert_spatial}, and, as noted in the 1996 revision of his book, become more important as audio spatialisation and entertainment technology attempts to induce auditory events that imply three-dimensional audio spaces.

\subsubsection{From two to three}

\begin{quotation}
    The external ears superimpose linear distortions on the incoming signals, which, in each case, are specific for the direction of incidence of the sound wave and the source distance.
    In this way, spatial information is encoded into the signals that are received by the eardrums~\citep{blauert_spatial}.
\end{quotation}

\citet{roginska2017immersive} note that:~\textquote{the word `binaural' refers, at the most basic level, to hearing with two ears, but it later came to include all the spatial cues from the ears, head, and body of a listener}.
Binaural recordings can therefore refer to the practice of capturing sounds that incorporate human physiology.
This is executed with dummy mannikin heads with microphones placed inside the ears so that sound entering them are affected by the `blocking' nature of the head;
developments in this technology rapidly sped up throughout the 20th century~\citep{binaural_paul}.
While other forms of spatial representation were developed in this time period~\citep{gerzon_periphony, noisternig_ambisonic, wave_field}, technology that considers the physical and physiological factors in human listening when attempting to induce auditory events that feature sound localization.

\citet{roginska2017immersive} identify that while capturing binaural audio is relatively easy, realizing the same effect through post-recording production is considerably harder and poses the challenge of modelling the human spatialization facility.

\subsubsection{Getting the head in the game}

The~\glsaccessfirst{hrtf} can be described as a representation of the perceptual cues that facilitate human sound localization as a sound propagates from its source to the human ear~\citep{Suzuki2011}.
This modelling of the human sound localization facility allows for this~\gls{hrtf} to be applied to a sound before reaching the human eardrum~\citep{roginska2017immersive}.
It is with this technology that more and more modern entertainment systems begin to localize sounds~\citep{blauert_spatial, HONDA2007, roginska2017immersive, Suzuki2011, Xie2013, ps5_audio, soundscape_design} during audio playback.

There are many software systems, toolkits, and frameworks that have been developed to allow engineers to build software that can utilise~\glspl{hrtf} and apply them to monophonic recordings~\citep{3d_tune_in, resonance}.
It is through these technologies that many video game systems such as the~\gls{ps5} are able to provide immersive 3D audio experiences.
In commercial systems such as these, consideration must also be applied to the selection of the~\gls{hrtf} that are used.
While each person's experience of sound is as individual as they are, capturing the~\gls{hrtf} of each individual who engages with the product is not yet feasible due to the highly involved and costly process of capturing them.
Considerable research has been done in order to develop and produce~\gls{hrtf} databases that appeal to a wide variety of subjects, taking into account individual and non-individual~\glspl{hrtf}~\citep{armstrong_}.
It is common practice to have entertainment systems contain multiple~\gls{hrtf} options to choose from when setting up that system's spatial audio capabilities~\citep{shukla2018user}.

\subsection{Web Audio}\label{subsec:web-audio}

Audio-visual media on the internet is extremely widespread and its delivery takes a myriad of forms~\citep{Bruegger2018}.
The means by which audio is delivered to users on the internet is most frequently through a web audio~\gls{api}, the most common of which is the one developed by Mozilla~\citep{w3c_audio_api, mdn_audio_api}.
One of the major benefits of utilising web technology in combination with audio technology is that it allows developers and those who wish to present audio to an audience to do so with a rich toolset of graphical libraries that are easily accessed through a web browser~\citep{Pauwels2018pywebaudioplayerBT};
this is most frequently seen in commercial usage through web audio players such as Spotify and SoundCloud as a natural evolution of the radio broadcasting format~\citep{Bottomley2020}.
Web frameworks such as React.js\footnote{\citep{Minnick2022}}, Flask\footnote{\citep{Zhai2022}}, and Django\footnote{\citep{Pauwels2018pywebaudioplayerBT}} are all capable of handling and displaying audio from a web page.

Audio delivery is primarily executed through the downloading and playing of a static file or as a packet stream from a server-based audio file source;
however, more recently web audio can be delivered peer-to-peer in real-time through such technologies like WebRTC~\citep{webrtc, Garcia2019}.

