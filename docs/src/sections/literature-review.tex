%! Author = cspiller
%! Date = 17/11/2022

\thispagestyle{plain}
\newpage
\section{Literature Review}\label{sec:literature-review}

\normalsize

As outlined in~\ref{subsec:aims}, this project attempts to engage with cloud infrastructure, spatial audio processing, and web development frameworks.
To approach these topics thoroughly, this report performs a review of pertinent literature.
In doing so, this report considers and incorporates existing ideas in these fields while providing a strong foundation to answer the research questions listed in~\ref{subsec:research-questions}.

\subsection{Audio Spatialisation}\label{subsec:audio-spatialisation}
\subsubsection{Origins}

\citet{blauert_spatial} notes in their seminal text,~\textit{\usebibentry{blauert_spatial}{title}}, that: \textquote{human beings are primarily visually-orientated}, and that the other senses are less developed in comparison.
This difference has been mirrored in the history of scientific research.
\citet{wade_binaural} note that research in binaural hearing was developed later than binocular vision partially due to the difficulty in controlling the audio stimuli in experiments.
It was only later on that the concept of distinction between the~\textit{sound event} and the~\textit{auditory event} as influenced by binaural hearing became prevalent.
\citet{blauert_spatial} explains that this distinction informs the practice of audio replication analogous to its originating sound event:

\begin{quotation}~
    The telecommunications engineer, of course, is especially interested in just those cases in which the positions of the sound source, and the auditory event do not coincide.
    The telecommunications engineer seeks to reproduce the auditory events that occur at the point where a recording or transmission originates, using the smallest possible number of sound sources (e.g., loudspeakers)~\citep{blauert_spatial}.
\end{quotation}

The patent filed in 1958 by Alan Dower Blumlein~\footnote{\citep{blumlein-patent}} details an early stereophonic system, which exploits the human sound localization ability so that entertainment experiences might be enhanced.
Blumlein observed that in film theatres there was a certain level of cognitive dissonance whereby the actor’s voices sounded like they were coming from a different location than where they appeared on the screen~\citep{alexander_blumlein}.
This patent, in response, specifically outlines methods for introducing stereophonic audio to sound film as a means of increasing the perceived~\textquotesingle{quality} of the entertainment experience.
Blumlein acknowledges that human binaural hearing is responsible for the ability to localize sound, and his patent is an example of how one might induce an auditory event that exhibits spatialisation on the horizontal plane~\citep{blumlein-patent}.


The patent marked an improvement in the way that auditory events might be replicated by introducing this form of spatialisation, and the vestiges of Blumlein`s ideas can be observed in modern spatial audio techniques~\citep{politis_spatial, beyer_acoustics}.
What is more salient, however, is that this patent also shows how the human’s ability to locate sound sources is not solely dependent on the nature of the sound itself, as identified by~\citet{blauert_spatial}.
Rather, it is the case that many other physiological and psychological factors are at play.

\subsubsection{From Two To Three}

